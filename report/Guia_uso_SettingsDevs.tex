\documentclass[]{article}

\usepackage[T1]{fontenc}
\usepackage{graphicx}
\usepackage{tabularborder}
\usepackage{booktabs}
\usepackage{tabularx}
\usepackage{float}
\usepackage{tikz}
\usepackage{placeins}
\usepackage{caption}
\usetikzlibrary{shapes}
\usetikzlibrary{fit}
\usepackage{fancyhdr}
\pagestyle{fancy}
\fancyhf{}
\renewcommand{\headrulewidth}{0pt}
\renewcommand{\footrulewidth}{0pt}
\fancyfoot[C]{\thepage}
\usepackage{graphicx}
\usepackage{amsmath}
\usepackage{pgfplots}
\usepackage{float}
%\usepackage[margin=1in]{geometry}
\usepackage[spanish]{babel}
\usepackage{amsmath}
\usepackage{amssymb}
\usepackage{caption}
\usepackage{subcaption}
\usepackage{graphicx}

\pgfplotsset{compat=1.16}
\title{\textbf{Settings Devs - Guía de uso}}
\author{Lucio Mansilla}
\date{\today}

\begin{document}

\maketitle
\begin{abstract}
    Este documento sirve como una guía de usuario para un software externo con interfaz gráfica de usuario que facilita la manipulación de ciertas configuraciones del Proyecto PowerDevs. El software proporciona una interfaz intuitiva y fácil de usar para crear y exportar el estado inicial y el esquema PDM, así como configurar PowerDEVS y restablecerlo a su versión original.
    \end{abstract}
    
    \section{Introducción}
    El software PowerDevs es una herramienta de configuración para el Proyecto PowerDevs. Proporciona una interfaz de usuario para generar un estado inicial, exportarlo a un archivo, crear un esquema PDM y restablecer a la versión original de PowerDEVS.
    
    \section{Creando un estado inicial}
    
    Para crear un estado inicial, debemos ingresar al tab "Crear Estado Inicial" (ver Figura X). Aquí, especificamos el número de filas y columnas para el estado inicial mediante los campos "Filas (N)" y "Columnas (N)".
    
    \begin{figure}[H]
    \centering
    %\includegraphics[width=0.75\textwidth]{crear_estado_inicial.png}
    \caption{Crear estado inicial en PowerDEVS}
    \label{fig:crear_estado_inicial}
    \end{figure}
    
    Hacemos clic en el botón "Crear cuadrícula". Aparecerá una cuadrícula de botones blancos en la ventana. Cada botón puede ser clickeado para cambiar su estado, alternando entre blanco (0) y negro (1).
    
    \section{Exportar el estado inicial}
    
    Una vez que hemos establecido el estado inicial, podemos exportarlo a un archivo de texto. Para hacer esto, hacemos clic en el botón "Exportar txt". Se nos pedirá que seleccionemos un nombre de archivo y una ubicación para guardar el archivo.
    
    \begin{figure}[H]
    \centering
    %\includegraphics[width=0.75\textwidth]{exportar_estado_inicial.png}
    \caption{Exportar estado inicial en PowerDEVS}
    \label{fig:exportar_estado_inicial}
    \end{figure}
    
    \section{Creando un esquema PDM}
    
    En la pestaña "Crear Esquema PDM" (ver Figura Y), especificamos el número de filas y columnas para el esquema PDM mediante los campos "Filas (N)" y "Columnas (N)". Hacemos clic en el botón "Generar PDM" para crear el esquema PDM.
    
    \begin{figure}[H]
    \centering
   % \includegraphics[width=0.75\textwidth]{crear_esquema_pdm.png}
    \caption{Crear esquema PDM en PowerDEVS}
    \label{fig:crear_esquema_pdm}
    \end{figure}
    
    \section{Configurando PowerDEVS}
    
    En la pestaña "Configuración del Proyecto", tenemos dos botones. Al hacer clic en "Setear PowerDEVS", se ejecuta el script build.sh para configurar PowerDEVS. Al hacer clic en "Restablecer a versión Original", se ejecuta el script restore.sh para restablecer PowerDEVS a su versión original.
    
    \begin{figure}[H]
    \centering
   % \includegraphics[width=0.75\textwidth]{configuracion_del_proyecto.png}
    \caption{Configuración del Proyecto en PowerDEVS}
    \label{fig:configuracion_del_proyecto}
    \end{figure}
    
    \section{Ayuda}
    
    En la pestaña "Ayuda", hay dos botones. Al hacer clic en "Guía de uso", se abre el navegador predeterminado del usuario y se dirige a la guía de usuario en línea. Al hacer clic en "Acerca de", se abre una nueva ventana con información adicional sobre el software.
    
    \begin{figure}[H]
    \centering
    %\includegraphics[width=0.75\textwidth]{ayuda.png}
    \caption{Ayuda en PowerDEVS}
    \label{fig:ayuda}
    \end{figure}
    
    \section{Conclusión}
    
    Con este software de configuración, los usuarios pueden crear y exportar el estado inicial, generar un esquema PDM y configurar PowerDEVS de manera fácil y conveniente. Esperamos que esta guía sea de utilidad para usar este software.
    
    \end{document}